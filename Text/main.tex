%!TEX TS-program = xelatex

% Шаблон документа LaTeX создан в 2018 году
% Алексеем Подчезерцевым
% В качестве исходных использованы шаблоны
% 	Данилом Фёдоровых (danil@fedorovykh.ru) 
%		https://www.writelatex.com/coursera/latex/5.2.2
%	LaTeX-шаблон для русской кандидатской диссертации и её автореферата.
%		https://github.com/AndreyAkinshin/Russian-Phd-LaTeX-Dissertation-Template

\documentclass[a4paper,14pt]{article}

\input{data/preambular.tex}
\begin{document} % конец преамбулы, начало документа
\input{data/title.tex}
\tableofcontents
\pagebreak

\section{Анализ предметной области}

База данных создаётся для информационного обслуживания системы онлайн курсов.
БД должна содержать данные о пользователях, курсах, материалов курсов (лекции и тесты), а так же результаты прохождения курсов.
В соответствии с предметной областью система строится с учётом следующих особенностей:

\begin{itemize}
	\item Каждый пользователь может пройти любой курс и каждый курс может пройти много пользователей;
	\item В каждом курсе может быть много блоков, блок может быть только в одном курсе;
	\item Каждый блок состоит из лекционных материалов и проверочных работ;
	\item Преподаватель может создавать лекционные материалы, а ученик может их прочитать;
	\item Преподаватель может создавать и проверять проверочные работы, ученик может несколько раз решать проверочные задания, а ассистент может проверять работы учеников;
	\item Администратор системы может создавать курсы и назначать преподавателей;
	\item Преподаватель может назначать ассистентов на курс;
	\item На каждом курсе может быть неограниченное число преподавателей, ассистентов и учеников, при этом каждый пользователь может учавствовать только с одной ролью.	
\end{itemize}

Сущности предметной области:

\begin{enumerate}
	\item \textbf{Пользователь}. Атрибуты: ФИО, логин, пароль, пол, почта;
	\item \textbf{Курс}. Атрибуты: названия, категория, видимость;
	\item \textbf{Блок}. Атрибуты: тема, видимость;
	\item \textbf{Лекция}. Атрибуты: название, содержание, продолжительность, видимость;
	\item \textbf{Проверочный материал}. Атрибуты: название, максимальный балл, продолжительность, видимость;
	\item \textbf{Решение}. Атрибуты: содержание, оценка.
\end{enumerate}

Исходя из выявленных сущностей, построим ER–диаграмму (Рис. \ref{img:db_ER}). 

\begin{figure}[H]
	\centering		
	\includegraphics[width=\linewidth]{schemas/ER}
	\caption{ER диаграмма}\label{img:db_ER}
\end{figure}

\section{Анализ информационных задач и круга пользователей системы}

Определим группы пользователей, их основные задачи и запросы к БД:

\begin{enumerate}
	\item Администратор ресурса
	\begin{itemize}
		\item Регистрация новых пользователей;
		\item Создание новых курсов;
		\item Привязка преподавателей к курсу;
		\item Управление платформой и курсами.
	\end{itemize}

	\item Преподаватель курса
	\begin{itemize}
		\item Регистрация и приглашение новых учеников и ассистентов на курс;
		\item Создание и редактирование информации о курсе (блоки, лекции, проверочные работы);
		\item Просмотр прогресса учеников по курсу;
		\item Проверка работ учащихся курса.
	\end{itemize}

	\item Ассистент курса
	\begin{itemize}
		\item Проверка работ учеников на подконтрольных курсов;
		\item Просмотр результатов собственной проверки.
	\end{itemize}

	\item Ученик
	\begin{itemize}
		\item Запись на существующие курсы;
		\item Чтение и фиксация прогресса по лекциям;
		\item Доступ к материалам проверочных работ;
		\item Многократное выполнение и отправка попыток на проверку;
		\item Доступ к результатам проверки.
	\end{itemize}
\end{enumerate}

\section{Определение требований к операционной обстановке}
На основе результатов анализа предметной области можно приблизительно оценить объём памяти, требуемой для хранения данных.
Примем ориентировочно, что:
\begin{itemize}
	\item В системе зарегистрировано 1000 пользователей (по 0.25К на запись);
	\item В системе создано 10 курсов, каждый из которых состоит в среднем из 32 блоков (0.2К на информацию о курсе);
	\item Каждый блок состоит из 5 лекционных материалов и 2 проверочных работ (0.1К на информацию о блоке);
	\item Каждая лекция содержит 5К текстовой информации;
	\item Каждый проверочный материал занимает 1К;
	\item Ученик записан в среднем на 2 курса;
	\item Ученик читает все лекции и решает в среднем 2 раза каждое задание (0.1К на чтение лекции и 3К на попытку решения).
\end{itemize}

Тогда объем памяти, занимаемый базой данных будет примерно равен:
\begin{multline*}
2 \times( 10 \times ((32 \times (5 \times 5K + 2 \times 1K) + 0.1K) + 0.2K) + \\ 
+ 1000 \times (2 \times 32 \times (5 \times 0.1K + 2 \times 2 \times 3K) + 0.25K)) = 1617786K	\approx 1.5G
\end{multline*}

\section{Выбор СУБД и других программных средств}
Будем писать на PostgreSQL.

\section{Преобразование ER–диаграммы в схему базы данных}

\begin{figure}[H]
	\centering		
	\includegraphics[width=\linewidth]{schemas/RDB}
	\caption{Схема базы данных}\label{img:RDB}
\end{figure}

\subsection{Составление реляционных отношений}

%1. Пользователь. Атрибуты: ФИО, логин, пароль, пол, почта;

Потенциальными ключами отношения ПОЛЬЗОВАТЕЛЬ являются поля логин и адрес электронной почты. Все они занимают достаточно много места. Введём суррогатный первичный ключ Номер пользователя.

\begin{table}[H]
	\begin{flushleft} 
		\tablecaption{\label{tab:UserV1} Схема отношения ПОЛЬЗОВАТЕЛЬ (User) }
	\end{flushleft}
\begin{tabular}{|l|l|c|l|}
	\hline
	Содержание поля        & Имя поля & Тип, длина & Примечания                     \\ \hline
	Номер пользователя     &          &    N(4)    & суррогатный первичный ключ     \\ \hline
	Фамилия, имя, отчество &          &   V(100)   & обязательное поле              \\ \hline
	Дата рождения          &          &    Date    & обязательное поле              \\ \hline
	Пол                    &          &    C(1)    & обязательное поле, 'м' или 'ж' \\ \hline
	Логин                  &          &   V(30)    & обязательное поле              \\ \hline
	Почта                  &          &   V(30)    & обязательное поле              \\ \hline
	Пароль                 &          &   V(30)    & обязательное поле              \\ \hline
\end{tabular}
\end{table}

%2. Курс. Атрибуты: названия, категория, видимость;

Потенциальным ключом отношения КУРС является поле название. Оно занимает достаточно много места. Введём суррогатный первичный ключ номер курса.

\begin{table}[H]
	\begin{flushleft} 
		\tablecaption{\label{tab:CourseV1} Схема отношения КУРС (Course) }
	\end{flushleft}
	\begin{tabular}{|l|l|c|l|}
		\hline
		Содержание поля & Имя поля & Тип, длина & Примечания                       \\ \hline
		Номер курса     &          &    N(4)    & суррогатный первичный ключ       \\ \hline
		Название        &          &    V(100)    & обязательное поле                \\ \hline
		Категории       &          &    V(100)    & обязательное поле                \\ \hline
		Видимость       &          &    BOOL    & обязательное поле, default FALSE \\ \hline
	\end{tabular}
\end{table}

%3. Блок. Атрибуты: тема, видимость;

У отношения БЛОК нет потенциального ключа, поэтому введём суррогатный первичный ключ номер блока.

\begin{table}[H]
	\begin{flushleft} 
		\tablecaption{\label{tab:BlockV1} Схема отношения КУРС (Block) }
	\end{flushleft}
	\begin{tabular}{|l|l|c|l|}
		\hline
		Содержание поля & Имя поля & Тип, длина & Примечания                       \\ \hline
		Номер блока     &          &    N(4)    & суррогатный первичный ключ       \\ \hline
		Тема            &          &    V(100)    & обязательное поле                \\ \hline
		Видимость       &          &    BOOL    & обязательное поле, default FALSE \\ \hline
	\end{tabular}
\end{table}

%4. Лекция. Атрибуты: название, содержание, продолжительность, видимость;

У отношения ЛЕКЦИЯ нет потенциального ключа, поэтому введём суррогатный первичный ключ номер лекции.

\begin{table}[H]
	\begin{flushleft} 
		\tablecaption{\label{tab:LectureV1} Схема отношения ЛЕКЦИЯ (Lecture) }
	\end{flushleft}
	\begin{tabular}{|l|l|c|l|}
		\hline
		Содержание поля   & Имя поля & Тип, длина & Примечания                       \\ \hline
		Номер лекции      &          &    N(4)    & суррогатный первичный ключ       \\ \hline
		Название          &          &    V(100)    & обязательное поле                \\ \hline
		Содержание        &          &    TEXT    &                                  \\ \hline
		Продолжительность &          &    INTERVAL    &                                  \\ \hline
		Видимость         &          &    BOOL    & обязательное поле, default FALSE \\ \hline
	\end{tabular}
\end{table}
%5. Проверочный материал. Атрибуты: название, максимальный балл, продолжительность, видимость;

У отношения ПРОВЕРОЧНЫЙ МАТЕРИАЛ нет потенциального ключа, поэтому введём суррогатный первичный ключ номер проверочного материала.

\begin{table}[H]
	\begin{flushleft} 
		\tablecaption{\label{tab:TestMaterialV1} Схема отношения ПРОВЕРОЧНЫЙ МАТЕРИАЛ (TestMaterial) }
	\end{flushleft}
\begin{adjustbox}{width=\linewidth}
	\begin{tabular}{|l|l|c|l|}
		\hline
		Содержание поля              & Имя поля & Тип, длина & Примечания                       \\ \hline
		Номер проверочного материала &          &    N(4)    & суррогатный первичный ключ       \\ \hline
		Название                     &          &   V(100)   & обязательное поле                \\ \hline
		Задание                      &          &    TEXT    & обязательное поле                \\ \hline
		Максимальный балл            &          &    N(4)    & обязательное поле                \\ \hline
		Продолжительность            &          &  INTERVAL  &                                  \\ \hline
		Видимость                    &          &    BOOL    & обязательное поле, default FALSE \\ \hline
	\end{tabular}
\end{adjustbox}
\end{table}

%6. Решение. Атрибуты: содержание, оценка.

У отношения РЕШЕНИЕ нет потенциального ключа, поэтому введём суррогатный первичный ключ номер решения.

\begin{table}[H]
	\begin{flushleft} 
		\tablecaption{\label{tab:DecisionV1} Схема отношения РЕШЕНИЕ (Decision) }
	\end{flushleft}
	\begin{tabular}{|l|l|c|l|}
		\hline
		Содержание поля & Имя поля & Тип, длина & Примечания                       \\ \hline
		Номер решения   &          &    N(4)    & суррогатный первичный ключ       \\ \hline
		Содержание      &          &    TEXT    & обязательное поле                \\ \hline
		Оценка          &          &    N(4)    &                                  \\ \hline
		Дата сдачи      &          &    Date    & обязательное поле, default NOW() \\ \hline
	\end{tabular}
\end{table}

%7. Участие. Атрибуты: роль, номер пользователь, номер курса.

У отношения УЧАСТИЕ нет потенциального ключа, поэтому введём суррогатный первичный ключ номер решения.

\begin{table}[H]
	\begin{flushleft} 
		\tablecaption{\label{tab:ParticipationV1} Схема отношения УЧАСТИЕ (Participation) }
	\end{flushleft}
	\begin{tabular}{|l|l|c|l|l|}
		\hline
		Содержание поля & Имя поля & Тип, длина & \multicolumn{2}{l|}{Примечания} \\ \hline
		Роль &  & С(10) & \multicolumn{2}{l|}{обязательное поле} \\ \hline
		Номер пользователя &  & N(4) & \begin{tabular}[c]{@{}l@{}}
			  внешний ключ   \\
			(к User)
		\end{tabular} & \multirow{2}{*}{\begin{tabular}[c]{@{}l@{}}составной\\ первичный \\ ключ\end{tabular}} \\ \cline{1-4}
		Номер курса &  & N(4) & \begin{tabular}[c]{@{}l@{}}внешний ключ \\ (к Course)\end{tabular} &  \\ \hline
	\end{tabular}
\end{table}


%\newpage 
%\renewcommand{\refname}{{\normalsize СПИСОК ИСПОЛЬЗОВАННЫХ ИСТОЧНИКОВ}} 
%\centering 
%\begin{thebibliography}{9} 
%	\addcontentsline{toc}{section}{\refname} 
%	\bibitem{sql} Beaulieu A. Learning SQL: Master SQL Fundamentals. – " O'Reilly Media, Inc.", 2009.
%	
%\end{thebibliography}

\end{document} % конец документа