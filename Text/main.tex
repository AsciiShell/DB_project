%!TEX TS-program = xelatex

% Шаблон документа LaTeX создан в 2018 году
% Алексеем Подчезерцевым
% В качестве исходных использованы шаблоны
% 	Данилом Фёдоровых (danil@fedorovykh.ru) 
%		https://www.writelatex.com/coursera/latex/5.2.2
%	LaTeX-шаблон для русской кандидатской диссертации и её автореферата.
%		https://github.com/AndreyAkinshin/Russian-Phd-LaTeX-Dissertation-Template

\documentclass[a4paper,14pt]{article}

\input{data/preambular.tex}
\begin{document} % конец преамбулы, начало документа
\input{data/title.tex}
\tableofcontents
\pagebreak

\section{Анализ предметной области}

База данных создаётся для информационного обслуживания системы онлайн курсов.
БД должна содержать данные о пользователях, курсах, материалов курсов (лекции и тесты), а так же результаты прохождения курсов.
В соответствии с предметной областью система строится с учётом следующих особенностей:

\begin{itemize}
	\item Каждый пользователь может пройти любой курс и каждый курс может пройти много пользователей;
	\item В каждом курсе может быть много блоков, блок может быть только в одном курсе;
	\item Каждый блок состоит из лекционных материалов и проверочных работ;
	\item Преподаватель может создавать лекционные материалы, а ученик может их прочитать;
	\item Преподаватель может создавать и проверять проверочные работы, ученик может несколько раз решать проверочные задания, а ассистент может проверять работы учеников;
	\item Администратор системы может создавать курсы и назначать преподавателей;
	\item Преподаватель может назначать ассистентов на курс;
	\item На каждом курсе может быть неограниченное число преподавателей, ассистентов и учеников.	
\end{itemize}

Сущности предметной области:

\begin{enumerate}
	\item \textbf{Пользователь}. Атрибуты: ФИО, логин, пароль, пол, почта;
	\item \textbf{Курс}. Атрибуты: названия, категория, видимость;
	\item \textbf{Блок}. Атрибуты: тема, видимость;
	\item \textbf{Лекция}. Атрибуты: название, содержание, продолжительность, видимость;
	\item \textbf{Проверочный материал}. Атрибуты: название, максимальный балл, продолжительность, видимость;
	\item \textbf{Решение}. Атрибуты: содержание, оценка.
\end{enumerate}

%\newpage 
%\renewcommand{\refname}{{\normalsize СПИСОК ИСПОЛЬЗОВАННЫХ ИСТОЧНИКОВ}} 
%\centering 
%\begin{thebibliography}{9} 
%	\addcontentsline{toc}{section}{\refname} 
%	\bibitem{sql} Beaulieu A. Learning SQL: Master SQL Fundamentals. – " O'Reilly Media, Inc.", 2009.
%	
%\end{thebibliography}

\end{document} % конец документа