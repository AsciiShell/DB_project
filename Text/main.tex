%!TEX TS-program = xelatex

% Шаблон документа LaTeX создан в 2018 году
% Алексеем Подчезерцевым
% В качестве исходных использованы шаблоны
% 	Данилом Фёдоровых (danil@fedorovykh.ru) 
%		https://www.writelatex.com/coursera/latex/5.2.2
%	LaTeX-шаблон для русской кандидатской диссертации и её автореферата.
%		https://github.com/AndreyAkinshin/Russian-Phd-LaTeX-Dissertation-Template

\documentclass[a4paper,14pt]{article}

\input{data/preambular.tex}
\begin{document} % конец преамбулы, начало документа
\input{data/title.tex}
\tableofcontents
\pagebreak

\section{Анализ предметной области}

База данных создаётся для информационного обслуживания системы онлайн курсов.
БД должна содержать данные о пользователях, курсах, материалов курсов (лекции и тесты), а так же результаты прохождения курсов.
В соответствии с предметной областью система строится с учётом следующих особенностей:

\begin{itemize}
	\item Каждый пользователь может пройти любой курс и каждый курс может пройти много пользователей;
	\item В каждом курсе может быть много блоков, блок может быть только в одном курсе;
	\item Каждый блок состоит из лекционных материалов и проверочных работ;
	\item Преподаватель может создавать лекционные материалы, а ученик может их прочитать;
	\item Преподаватель может создавать и проверять проверочные работы, ученик может несколько раз решать проверочные задания, а ассистент может проверять работы учеников;
	\item Администратор системы может создавать курсы и назначать преподавателей;
	\item Преподаватель может назначать ассистентов на курс;
	\item На каждом курсе может быть неограниченное число преподавателей, ассистентов и учеников.	
\end{itemize}

Сущности предметной области:

\begin{enumerate}
	\item \textbf{Пользователь}. Атрибуты: ФИО, логин, пароль, пол, почта;
	\item \textbf{Курс}. Атрибуты: названия, категория, видимость;
	\item \textbf{Блок}. Атрибуты: тема, видимость;
	\item \textbf{Лекция}. Атрибуты: название, содержание, продолжительность, видимость;
	\item \textbf{Проверочный материал}. Атрибуты: название, максимальный балл, продолжительность, видимость;
	\item \textbf{Решение}. Атрибуты: содержание, оценка.
\end{enumerate}

Исходя из выявленных сущностей, построим ER–диаграмму (Рис. \ref{img:db_ER}). 

\begin{figure}[H]
	\centering		
	\includegraphics[width=\linewidth]{schemas/ER}
	\caption{ER диаграмма}\label{img:db_ER}
\end{figure}

\section{Анализ информационных задач и круга пользователей системы}

Определим группы пользователей, их основные задачи и запросы к БД:

\begin{enumerate}
	\item Администратор ресурса
	\begin{itemize}
		\item Регистрация новых пользователей;
		\item Создание новых курсов;
		\item Привязка преподавателей к курсу;
		\item Управление платформой и курсами.
	\end{itemize}

	\item Преподаватель курса
	\begin{itemize}
		\item Регистрация и приглашение новых учеников и ассистентов на курс;
		\item Создание и редактирование информации о курсе (блоки, лекции, проверочные работы);
		\item Просмотр прогресса учеников по курсу;
		\item Проверка работ учащихся курса.
	\end{itemize}

	\item Ассистент курса
	\begin{itemize}
		\item Проверка работ учеников на подконтрольных курсов;
		\item Просмотр результатов собственной проверки.
	\end{itemize}

	\item Ученик
	\begin{itemize}
		\item Запись на существующие курсы;
		\item Чтение и фиксация прогресса по лекциям;
		\item Доступ к материалам проверочных работ;
		\item Многократное выполнение и отправка попыток на проверку;
		\item Доступ к результатам проверки.
	\end{itemize}
\end{enumerate}

\section{Определение требований к операционной обстановке}
На основе результатов анализа предметной области можно приблизительно оценить объём памяти, требуемой для хранения данных.
Примем ориентировочно, что:
\begin{itemize}
	\item В системе зарегистрировано 1000 пользователей (по 0.25К на запись);
	\item В системе создано 10 курсов, каждый из которых состоит в среднем из 32 блоков (0.2К на информацию о курсе);
	\item Каждый блок состоит из 5 лекционных материалов и 2 проверочных работ (0.1К на информацию о блоке);
	\item Каждая лекция содержит 5К текстовой информации;
	\item Каждый проверочный материал занимает 1К;
	\item Ученик записан в среднем на 2 курса;
	\item Ученик читает все лекции и решает в среднем 2 раза каждое задание (0.1К на чтение лекции и 3К на попытку решения).
\end{itemize}

Тогда объем памяти, занимаемый базой данных будет примерно равен:
\begin{multline*}
2 \times( 10 \times ((32 \times (5 \times 5K + 2 \times 1K) + 0.1K) + 0.2K) + \\ 
+ 1000 \times (2 \times 32 \times (5 \times 0.1K + 2 \times 2 \times 3K) + 0.25K)) = 1617786K	\approx 1.5G
\end{multline*}

\section{Выбор СУБД и других программных средств}
Будем писать на PostgreSQL.

\section{Преобразование ER–диаграммы в схему базы данных}

\begin{figure}[H]
	\centering		
	\includegraphics[width=\linewidth]{schemas/RDB}
	\caption{Схема базы данных}\label{img:RDB}
\end{figure}

%\newpage 
%\renewcommand{\refname}{{\normalsize СПИСОК ИСПОЛЬЗОВАННЫХ ИСТОЧНИКОВ}} 
%\centering 
%\begin{thebibliography}{9} 
%	\addcontentsline{toc}{section}{\refname} 
%	\bibitem{sql} Beaulieu A. Learning SQL: Master SQL Fundamentals. – " O'Reilly Media, Inc.", 2009.
%	
%\end{thebibliography}

\end{document} % конец документа